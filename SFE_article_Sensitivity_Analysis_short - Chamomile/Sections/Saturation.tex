\documentclass[../Article_Model_Parameters.tex]{subfiles}
\graphicspath{{\subfix{../Figures/}}}
\begin{document}
	
	\subsubsection{Uneven solute's distribution in the solid phase} \label{CH: Gamma_Function}
	
	Following the idea of the Broken-and-Intact Cell (BIC) model (\citet{Sovova2017}), the internal diffusion coefficient $D_i$ is consider to be a product of the reference value of $D_i^R$ and the exponential decay function $\gamma$, as given by Equation \ref{EQ: C_sat_function}.
	
	{\footnotesize
		\begin{equation}
			D_i = D_i^R \gamma(c_s) = D_i^R \exp \left( \Upsilon \left( 1-\cfrac{ c_s }{c_{s0}} \right) \right) \label{EQ: C_sat_function}
	\end{equation} }
	
	where the ${\color{black}\Upsilon}$ describe the curvature of the decay function. Equation \ref{Model_kinetic} describes the final form of the kinetic term
	
	{\footnotesize
		\begin{equation}
			\label{Model_kinetic}
			r_e = -\cfrac{D_i^R \gamma }{ \mu l^2 } \left( c_s  - \cfrac{\rho_s c_f }{ k_m \rho_f }  \right)
	\end{equation} }
	
	The $\gamma$-function limits the solute's availability in the solid phase. Similarly to the BIC model, the solute is assumed to be contained in the cells, a part of which is open because the cell walls were broken by grinding, and the rest remains intact. The diffusion of the solute from a particle's core takes more time than the diffusion of the solute close to the outer surface. The same idea can be represented by the decaying internal diffusion coefficient, where the decreasing term is a function of the solute concentration in the solid. 
	
	Alternatively, the decay function $\gamma$ can be consider with respect to the Shrinking Core model presented by \citet{Goto1996}, where the particle radius change as the amount of solute in the solid phase decrease. As the particle size decreases due to dissolution, the diffusion path increases, which makes the diffusion slower and reduces the value of a diffusion coefficient. This analogy can be applied to the Equation \ref{EQ: C_sat_function} to justify the application of varying diffusion coefficient.
	
\end{document}