\documentclass[]{article}

%opening
\title{Cost calculations}
\author{}

\begin{document}

\maketitle

\begin{abstract}
Summery of the cost calculations
\end{abstract}

\section{Intro}

The estimation of operating and capital costs is an important facet of process design and optimization. In the absence of firm bids or valid historical records, you can locate charts, tables, and equations that provide cost estimates from a wide variety of sources based on given values of the design variables.

\section{Cost index method}

When using purchase cost estimates from an earlier date, an estimate of the cost at the later date is made by multiplying the cost from an earlier date by the ratio of a cost index, $I$, at that later date to a base cost index, $I_{base}$, that corresponds to the date that applies to the purchase cost:

{\footnotesize
\begin{equation}
	Cost = Base~Cost \left(\frac{I}{I_{base}}\right)
	\end{equation}}

The indexes most commonly considered by chemical engineers are: The Chemical Engineering (CE) Plant Cost Index, The Marshall \& Swift (MS) Equipment Cost Index, The Nelson-Farrar (NF) Refinery Construction Cost Index and The Engineering News-Record (ENR) Construction Cost Index.

The CE and NF indexes pertain to the entire processing plant, taking into account labor and materials to fabricate the equipment, deliver it, and install it. However, the NF index is restricted to the petroleum industry whereas the CE index pertains to an average of all chemical processing industries. The ENR index, which is a more general index that pertains to the average of all industrial construction, is a composite of the costs of structural steel, lumber, concrete, and labor. The MS index pertains to an all-industry average equipment purchase cost. However, the MS index is accompanied by a more useful process industries average equipment cost index, averaged mainly for the chemicals, petroleum products, paper, and rubber industries. The CE and NF indexes also provide cost indexes for only the purchase cost of several categories of processing equipment, including heat exchangers, pumps and compressors, and machinery

\section{Compressor/Pump}

\section{Heat exchanger}

\section{Extractor}

\end{document}
