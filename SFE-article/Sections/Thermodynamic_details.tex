\documentclass[../Article_Model_Parameters.tex]{subfiles}
\graphicspath{{\subfix{../Figures/}}}
\begin{document}
	
	\label{CH: Thermodynamic_details}
	
	\subsubsection{Equation of state and properties of the fluid phase} \label{subsubsec: Equation of state}
	We consider equations of state in the general form ${\color{red}P}(t) {\color{orange}v_m}[T(t,z),P(t)] = {\color{orange}Z}{\color{magenta}R}{\color{blue}T}(t,z)$, where ${\color{magenta}v_m}$ denotes the molar volume of $\text{CO}_2$, ${\color{orange}Z}$ represents its compressibility factor, and $R$ is the universal gas constant. More specifically, we are interested in these equations because of the possibility to express the compressibility $Z$ as an explicit function of temperature and pressure. This is the case when $Z$ is obtained as one of the physically meaningful roots of a polynomial equation, like the Van der Waals's, Peng-Robinson's or Redlich–Kwong's equation of state. In this work the Peng-Robinson equation of state (\citet{Peng1976}) is used 
	
	In Peng-Robinson's equation fo state, the compressibility ${\color{orange}\overline{Z}}[{\color{blue}T}(t,z), {\color{red}P}(t)]$ solves the third-order polynomial equation given by equitation \ref{eq: Compressibility}. The polynomial equation can be solved by numerical methods like Newton-Raphson, or by applying Cardano formula. The Cardano formula is preferable in comparison to numerical root-finders, as it allows to obtain an analytical solution. The details of the Cardano formula are presented in appendix \ref{CH: Cardano}.
	
	{\footnotesize	
	\begin{align}\label{eq: Compressibility}
		&{\color{orange}Z}^3 - \left[1 - {\color{orange}B}\left[{\color{blue}T}(t,z), {\color{red}P}(t)\right]\right] {\color{orange}Z}^2 \\ \nonumber
		&+ \left[{\color{orange}A}\left[{\color{blue}T}(t,z), {\color{red}P}(t)\right] - 2{\color{orange}B}\left[{\color{blue}T}(t,z), {\color{red}P}(t)\right] - 3{\color{orange}B}\left[{\color{blue}T}(t,z), {\color{red}P}(t)\right]^2\right] {\color{orange}Z} = 0
	\end{align} }

	where ${\color{orange}A}\left[{\color{blue}T}(t,z), {\color{red}P}(t)\right]$ and ${\color{orange}B}\left[{\color{blue}T}(t,z), {\color{red}P}(t)\right]$ are functions of time and space defined on the attraction parameter, ${\color{orange}a}\left[{\color{blue}T}(t,z)\right] = {\color{magenta}a^c_{\text{CO}_2}}{\color{orange}\alpha}\left[{\color{blue}T}(t,z)\right]$ with ${\color{magenta}a^c_{\text{CO}_2}} \approx 0.45724 {{\color{magenta}R}^2{\color{magenta}T^c_{\text{CO}_2}}^2}/{{\color{magenta}P^c_{\text{CO}_2}}}$, and the repulsion parameter, ${\color{magenta}b_{\text{CO}_2}} \approx 0.07780 {{\color{magenta}R}{\color{magenta}T^c_{\text{CO}_2}}}/{{\color{magenta}P^c_{\text{CO}_2}}}$, both functions of the critical temperature ${\color{magenta}T^c_{\text{CO}_2}}$ and pressure ${\color{magenta}P^c_{\text{CO}_2}}$. Specifically, we have

	{\footnotesize
	\begin{subequations} \label{eq: PR_AB}
		\begin{align} 
			{\color{orange}A}\left[{\color{blue}T}(t,z), {\color{red}P}(t)\right]	& = \dfrac{{\color{orange}\alpha}\left[T\left(t,z\right)\right]{\color{magenta}a^c_{\text{CO}_2}}{\color{red}P}(t)}{{\color{magenta}R}^2{\color{blue}T}^2(t,z)};													\label{eq: PR_A}\\
			{\color{orange}B}\left[{\color{blue}T}(t,z), {\color{red}P}(t)\right]	& = \dfrac{{\color{magenta}b_{\text{CO}_2}}{\color{red}P}(t)}{{\color{magenta}R}{\color{blue}T}(t,z)}	\label{eq: PR_B}.
		\end{align}
	\end{subequations}}

	The quantity {\small${\color{orange}\alpha}\left[{\color{blue}T}\left(t,z\right)\right]= \left[1 + {\color{magenta}\kappa_{\text{CO}_2}} \left[1 - \sqrt{{\color{blue}T}\left(t,z\right) / {\color{magenta}T^c_{\text{CO}_2}}} \right] \right]^2$}, with constant ${\color{magenta}\kappa_{\text{CO}_2}} = 0.37464 + 1.54226 {\color{magenta}\omega_{\text{CO}_2}} - 0.26992 {\color{magenta}\omega_{\text{CO}_2}}^2$, is a dimensionless correction term defined on the acentric factor ${\color{magenta}\omega_{\text{CO}_2}} = 0.239$ of $\text{CO}_2$ molecules. \\
	
	By denoting the physical constants as {\small$\varphi_{Z} = \left(R,T^c_{\text{CO}_2},P^c_{\text{CO}_2},\kappa_{\text{CO}_2}\right)$}, we obtain a spatio-temporal representation of the compressibility ${\color{orange}\overline{Z}}\left[{\color{blue}T}(t,z), {\color{red}P}(t) \mid \varphi_{Z}\right]$ with its complete set of functional dependencies and parameters. \\
	
	\subsubsection{Density of the fluid phase} \label{subsubsec: Fluid density}
	%\noindent \textbf{Density of the fluid phase} \\
	
	The density $\rho_{\text{F}}$ of the fluid phase is assumed to be equal to the density of solvent, at given temperature and pressure. Because temperature $T(t,z)$ of the fluid phase is a modelled variable, we allow for the density to vary along the bed and in time. From an equation of state of the form ${\color{red}P}(t) {\color{orange}v_m}\left[T(t,z),P(t)\right] = {\color{orange}Z}{\color{magenta}R}{\color{blue}T}(t,z)$, we get
	
	{\footnotesize
	\begin{equation} \label{eq: Density}
		{\color{orange}\rho_{\text{F}}} \left[{\color{blue}T}(t,z),{\color{red}P}(t) \mid {\color{magenta}\varphi_{\rho_{\text{F}}}}\right] = \dfrac{{\color{red}P}(t) {\color{magenta}M_{\text{CO}_2}}}{{\color{magenta}R}{\color{blue}T}(t,z){\color{orange}\overline{Z}}\left[{\color{blue}T}(t,z),{\color{red}P}(t) \mid \varphi_{Z}\right]},
	\end{equation} }

	where ${\color{magenta}M_{\text{CO}_2}}$ denotes the molar mass of $\text{CO}_2$ and ${\color{orange}\overline{Z}}\left[{\color{blue}T}(t,z), {\color{red}P}(t)\right]$ is the compressibility factor that solves Eq. \eqref{eq: Compressibility}. The density of the fluid is thus a function of space and time, due to its dependence on temperature and pressure, and it is expressed in terms of the set of physical constants $\varphi_{\rho_\text{F}} = \left({\color{magenta}R}, {\color{magenta}M_{\text{CO}_2}}, {\color{magenta}T^c_{\text{CO}_2}}, {\color{magenta}P^c_{\text{CO}_2}}, {\color{magenta}\omega_{\text{CO}_2}}\right)$. \\
	
	\subsubsection{Heat capacity of the fluid phase} \label{subsubsec: Fluid heat capacity}
%	\noindent\textbf{Heat capacity of the fluid phase} \\
	
	The specific heat $C_p^{\text{F}}$ can be calculated from the equation of state, again under the assumption that the fluid phase consists of pure carbon dioxide and that the specific heat of real fluids can be calculated from an ideal contribution plus a residual term \citet{Pratt2001}. In the following, we report only the main steps in the derivation of ${\color{orange}C^{\text{CO}_2}_p}\left[{\color{blue}T}(t,z), {\color{red}P}(t)\right]$: 
	
	{\footnotesize
	\begin{subequations}\begin{align}
			{\color{orange}C^{\text{CO}_2}_v}\left[{\color{blue}T}(t,z), {\color{red}P}(t)\right] & = {\color{orange}C_v^{\text{I}}}\left[{\color{blue}T}(t,z),P(t)\right] + {\color{orange}C_v^{\text{R}}}\left[{\color{blue}T}(t,z), {\color{red}P}(t)\right] \label{eq: Cv_CO2}; \\ 
			{\color{orange}C^{\text{CO}_2}_p}\left[{\color{blue}T}(t,z), {\color{red}P}(t)\right] & = \underbrace{{\color{orange}C_p^{\text{I}}}\left[{\color{blue}T}(t,z),P(t)\right]}_{\text{Eq. } \eqref{eq: Cp_I}} + \underbrace{{\color{orange}C_p^{\text{R}}}\left[{\color{blue}T}(t,z), {\color{red}P}(t)\right]}_{\text{Eq. } \eqref{eq: Cp_R}}  \label{eq: Cp_CO2}.
	\end{align}\end{subequations} }

	${\color{orange}C^{\text{CO}_2}_v}\left[{\color{blue}T}(t,z), {\color{red}P}(t)\right]$ and ${\color{orange}C^{\text{CO}_2}_p}\left[{\color{blue}T}(t,z), {\color{red}P}(t)\right]$ are the specific heat of $\text{CO}_2$ at constant volume and pressure, respectively. ${\color{orange}C_v^{\text{I}}}\left[{\color{blue}T}(t,z), {\color{red}P}(t)\right]$ and ${\color{orange}C_p^{\text{I}}}\left[{\color{blue}T}(t,z), {\color{red}P}(t)\right]$, with ${\color{orange}C_p^{\text{I}}}({\color{blue}T}(t,z)) - {\color{orange}C_v^{\text{I}}}({\color{blue}T}(t,z)) = {\color{magenta}R}$, are the specific heat of an ideal gas at constant volume and pressure. ${\color{orange}C_v^{\text{R}}}\left[{\color{blue}T}(t,z), {\color{red}P}(t)\right]$ and ${\color{orange}C_p^{\text{R}}}({\color{blue}T}(t,z), {\color{red}P}(t))$ are the correction terms. \\
	
	For $CO_2$ ()\todo[fancyline]{Add REF}, we have the ideal gas contribution to the specific heat at constant $P(t)$, as function of $T(t,z)$,
	
	{\footnotesize
	\begin{equation}
		{\color{orange}C_p^{\text{I}}}\left[{\color{blue}T}(t,z),P(t)\right] = {\color{magenta}C_{P0}} + {\color{magenta}C_{P1}}T(t,z) + {\color{magenta}C_{P2}}{\color{blue}T}^2(t,z) + {\color{magenta}C_{P3}}{\color{blue}T}^3(t,z) \label{eq: Cp_I}
	\end{equation} }

	where the coefficients of the expansion are ${\color{magenta}C_{P0}} = 4.728$, ${\color{magenta}C_{P1}} = 1.75 \times 10^{-3}$, ${\color{magenta}C_{P2}} = -1.34 \times 10^{-5}$, and ${\color{magenta}C_{P3}} = 4.10 \times 10^{-9}$. For the correction term ${\color{orange}C_p^{\text{R}}}\left[{\color{blue}T}(t,z){\color{red}P}(t)\right]$ at constant pressure $P(t)$, we have
	
	{\footnotesize	
	\begin{align}
		&{\color{orange}C_p^{\text{R}}}\left[{\color{blue}T}(t,z), {\color{red}P}(t)\right] = \underbrace{{\color{orange}C_v^{\text{R}}}\left[{\color{blue}T}(t,z), {\color{red}P}(t)\right]}_{\text{Eq. }\eqref{eq: CvR}} \nonumber \\
		&+ {\color{blue}T}(t,z) \underbrace{\left(\cfrac{\partial {\color{red}P}(t)}{\partial {\color{blue}T}}\right)_{{\color{orange}v_m}(t,z)}}_{\text{Eq. } \eqref{eq: dPdT}} \underbrace{\left(\cfrac{\partial {\color{orange}v_m}\left[{\color{blue}T}(t,z),{\color{red}P}(t)\right]}{\partial {\color{blue}T}}\right)_{{\color{red}P}(t)}}_{\text{Eq. } \eqref{eq: dVdT}} - {\color{magenta}R}. \label{eq: Cp_R} 
	\end{align} }

	The braced terms are obtained from the chosen equation of state $P(t) V\left[T(t,z),P(t)\right] = Z\left[T(t,z),P(t)\right] R T(t,z)$. \\
	
	For the partial derivative of the volume with respect to temperature $T$ at constant pressure $P(t)$, we have

	{\footnotesize
	\begin{align}
		\left( \cfrac{\partial {\color{orange}v_m}\left[{\color{blue}T}(t,z), {\color{red}P}(t)\right]}{\partial {\color{blue}T}} \right)_{{\color{red}P}(t)} &= \cfrac{{\color{orange}Z}\left[{\color{blue}T}(t,z), {\color{red}P}(t)\right] {\color{magenta}R}}{{\color{red}P}(t)} \nonumber \\
		&+ \cfrac{{\color{magenta}R}{\color{blue}T}(t,z)}{{\color{red}P}(t)} \underbrace{\left( \cfrac{\partial {\color{orange}Z}\left[{\color{blue}T}(t,z), {\color{red}P}(t)\right]}{\partial {\color{blue}T}} \right)_{{\color{red}P}(t)}}_{\text{Eq. }\eqref{eq: dZdT}} \label{eq: dVdT}
	\end{align} }
 
	with partial derivative of the compressibility factor with respect to temperature $T$ at constant pressure $P(t)$

	{\footnotesize
	\begin{equation}
		\left(\cfrac{\partial{\color{orange}Z}\left[{\color{blue}T}(t,z), {\color{red}P}(t)\right]}{\partial {\color{blue}T}}\right)_{{\color{red}P}(t)} = \left(\cfrac{\partial \dfrac{{\color{red}P}(t) {\color{orange}v_m}\left[T(t,z),P(t)\right]}{{\color{magenta}R}{\color{blue}T}(t,z)}}{\partial {\color{blue}T}}\right)_{{\color{red}P}(t)} \label{eq: dZdT}
	\end{equation} }

	Similarly, for the partial derivative of the pressure with respect to temperature at constant volume, we have
	
	{\footnotesize
	\begin{equation}
		\left(\cfrac{\partial {\color{red}P}(t)}{\partial {\color{blue}T}}\right)_{{\color{orange}v_m}(t,z)} = \left( \cfrac{\partial{\dfrac{{\color{orange}Z}{\color{magenta}R}{\color{blue}T}(t,z)}{{\color{magenta}v_m}\left[T(t,z),P(t)\right]}}}{\partial{\color{blue}T}}\right)_{{\color{orange}v_m}(t,z)} \label{eq: dPdT}
	\end{equation} }
	
	The residual specific heat at constant volume is obtained, by definition, by using the residual internal energy
	
	{\footnotesize
	\begin{equation}
		{\color{orange}C_v^{\text{R}}}\left[{\color{blue}T}(t,z),{\color{red}P}(t)\right] = \left(\cfrac{\partial{\color{orange}U^{\text{R}}}\left[{\color{blue}T}(t,z),{\color{red}P}(t)\right]}{\partial {\color{blue}T}}\right)_{{\color{orange}v_m}(t,z)}. \label{eq: CvR}
	\end{equation} }
	
	By denoting the physical constants as $\varphi_{C_p^{\text{F}}} = ({\color{magenta}R}, {\color{magenta}M_{\text{CO}_2}},\\ {\color{magenta}T^c_{\text{CO}_2}}, {\color{magenta}P^c_{\text{CO}_2}}, {\color{magenta}\omega_{\text{CO}_2}})$, we get the spatio-temporal representation $C_p^{\text{F}}\left[T(t,z),P(t) \mid \varphi_{C_p^{\text{F}}}\right]$ of the specific heat of the fluid phase.
	
	\subsubsection{Departure functions for enthalpy calculations} \label{CH:Enthalpy}
	
	In thermodynamics, a departure function is a concept used to calculate the difference between a real fluid's thermodynamic properties and those of an ideal gas, given a specific temperature and pressure. Common departure functions include those for enthalpy, entropy, and internal energy. These functions are used to calculate extensive properties, which are properties computed as a difference between two states.
	
	For example, to evaluate the enthalpy change between two points, $h(V_1,T_1)$ and $h(V_2,T_2)$, we first calculate the enthalpy departure function between the initial volume $V_1$ and infinite volume at temperature $T_1$. We then add to that the ideal gas enthalpy change due to the temperature change from $T_1$ to $T_2$, and finally subtract the departure function value between the final volume $V_2$ and infinite volume.
	
	Departure functions are computed by integrating a function that depends on an equation of state and its derivative. The general form of the enthalpy equation is given by:
	
	{\footnotesize
		\begin{equation}
			\frac{h^{id}-h}{RT} =\int_{v_m}^{\infty }\left[T\left({\frac{\partial Z}{\partial T}}\right)_{v_m}\right]{\frac{dv_m}{v_m}} + 1-Z
		\end{equation}
	}
	
	Here, $h^{id}$ represents the enthalpy of an ideal gas, $h$ is the enthalpy of a real fluid, $R$ is the universal gas constant, $T$ is temperature, $v_m$ is the molar volume, and $Z$ is the compressibility factor.
	
	The integral in the equation is evaluated over the range of molar volumes from $v_m$ to infinity. The integral includes a term that depends on the derivative of the compressibility factor with respect to temperature, evaluated at the molar volume $v_m$. Finally, the term $1-Z$ is added to account for the deviation of the fluid's properties from those of an ideal gas.

	The Peng–Robinson equation of state relates the three interdependent state properties pressure $P$, temperature $T$, and molar volume $v_m$. From the state properties ($P$, $v_m$, $T$), one may compute the departure function for enthalpy per mole (denoted $h$) as presented by \citet{Gmehling2019} or \citet{Elliott2011}:
	
	{\footnotesize
		\begin{equation}
			h-h^{\mathrm {id} }=RT\left[T_{r}(Z-1)-2.078(1+\kappa ){\sqrt {\alpha }}\ln \left({\frac {Z+2.414B}{Z-0.414B}}\right)\right]
		\end{equation}
	}
	
	
\end{document}