\documentclass[../Article_Model_Parameters.tex]{subfiles}
\graphicspath{{\subfix{../Figures/}}}
\begin{document}
	
	It is assumed that the solvent is free of solute at the entrance of the extractor and that all the solid particles have the same initial solute content ${\color{black}c_s^0}$. It is considered that the initial temperature of the extractor in every place is the same and described by ${\color{black}h^0}$. 
	
	During the preparation period, the solute diffuses to the fluid phase in contact with the solid particles although the operating conditions are not obtained yet. Later, the solute in the fluid phase is partially moved (if the pressure increase in the system, the pump cause the movement of the fluid, even if the outlet valve is closed) to the region where there is no solid phase. As a result, the distribution of solutes mass in the fluid phase is assumed to not be uniform, and described by an arbitrary function $H=H(z)$. Some conclusions can be drawn from the analysis of the initial part of each yield curve. It can be noticed that each curve at the beginning has a curvature, which is not linear. In a general sense, it can be said that a quadratic function could approximate the initial part of each extraction curve. A function that, after integration, gives a quadratic-like result is a straight line. Based on that observation, the solute concentration in the fluid phase is assumed to be linearly distributed. The solute concentration is assumed to be zero at the outlet and reach the maximum at the beginning of the fixed bed. The graphical representation of the solute concentration in the fluid phase is shown in Figure \ref{fig: C_f_linear}. 
	
	\begin{figure}[!h]
		\centering
		\begin{tikzpicture}[scale=2]
			\draw[->] (-2,0) -- (2,0) node[right] {z}; 
			\draw[->] (0,-1.5) -- (0,2) node[above] {$H(z)$};
%			\draw[->] (-2,1) -- (2,1);
%			\draw[->] (1,-1.5) -- (1,2);
			\draw[blue] (-1,2)  -- (2,-1);
			
			\node at (0,0) {\textbullet};
			\node[] at (0.2,0.1) {$z=0$};
			
			\node at (1,0) {\textbullet};
			\node[] at (1.2,0.1) {$z=L$};
			
			\node at (0,1) {\textbullet};
			%\node[] at (0.3,1) {$H=m$};
			
			\node[black] at (1.3,0.5) {$H(z)=-m(z-b) + a$};
		\end{tikzpicture}
	\caption{The linear distribution of the solute concentration in the fluid phase}
	\label{fig: C_f_linear}
	\end{figure}

	As presented in Figure \ref{fig: C_f_linear}, the function $H$ is defined as
	
	{\footnotesize
	\begin{equation}
		H(z) = -m \left( z-b \right) + a
	\end{equation}
	}
	
	The function $H$ can be integrated over the integral from $a$ to $b$ to describe the total amount of solute in the fluid phase ($S$) in that interval. 
	
	{\footnotesize
		\begin{equation}
			\int_{a}^{b} H(z) dz = S 
		\end{equation}
	}

	If parameter $a$ describe the beginning of the fixed bed, then $a=0$. Similarly, $b$ can be defined as the end of the fixed bed, then $b=L$. On range from $a=0$ to $b=L$ the result of this integral is know, and it is equal to
	
	{\footnotesize
		\begin{equation}
			S = m_{fluid}^0 = \int_{a=0}^{b=L} -m \left(x-L\right) dz
			\label{EQ: C_f_linear_int}
		\end{equation}
	}

	where $m_{fluid}^0$ is the total mass of solute in the fluid phase and $N$ is the number of intervals between $a$ and $b$.
	The right-hand side of the above equation can be evaluated
	
	{\footnotesize
		\begin{align}
			\int_{a=0}^{b=L} -m \left(x-L\right) dz &= -m \left( \int_{a=0}^{b=L} z dz - \int_{a=0}^{b=L} L dz \right) \nonumber \\
													&= -m \left(\frac{z^2}{2} \biggr\rvert_0^L - Lz \biggr\rvert_0^L   \right) \nonumber \\
													&= -m \left( \frac{L^2}{2} - L^2 \right) \nonumber \\
													&=  m \frac{L^2}{2}
													\label{EQ: C_f_linear_after_int}
		\end{align}
	}

	The parameter $m$ can be obtained by equating the left-hand side of Equation \ref{EQ: C_f_linear_int} and the right-hand side of Equation \ref{EQ: C_f_linear_after_int}
	
	{\footnotesize
		\begin{equation}
			m_{fluid}^0 = m \frac{L^2}{2} \rightarrow m = \frac{2m_{fluid}^0}{L^2}
		\end{equation}
	}

	As can be seen from the above equation, the linearly distributed concentration of solute in the fluid phase can be fully determined if parameters $a$, $b$ and $m_{fluid}^0$ are know. $m_{fluid}^0$ can be obtained from the total mass of the solute in the system at the initial time $m_{total}^0$ and the initial mass ratio $\tau$.
	
	{\footnotesize
		\begin{equation*}
			\tau = \cfrac{\text{total mass of solute in the fluid phase}}{\text{total mass of solute in the system}} = \cfrac{m^0_{fluid}}{m^0_{total}}
		\end{equation*}
	}

	The initial conditions can be summarized as

	{\footnotesize
		\begin{subequations}
			\begin{align*}
				{\color{black}c_f}(t = 0, z) &= H(z)   \\
				{\color{black}c_s}(t = 0, z) &= {\color{black}c_{s0}} \\
				{\color{black}h}(t = 0, z) &= {\color{black}h_0}
			\end{align*}
	\end{subequations} }

\end{document}