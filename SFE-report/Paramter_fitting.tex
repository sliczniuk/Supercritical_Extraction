% ---------------------------------------------------------------
% Preamble
% ---------------------------------------------------------------
%\documentclass[a4paper,fleqn,longmktitle]{cas-sc}
\documentclass[a4paper,fleqn]{cas-dc}
%\documentclass[a4paper]{cas-dc}
%\documentclass[a4paper]{cas-sc}
% ---------------------------------------------------------------

% -------------------------------------------------------------------- 
% Packages
% --------------------------------------------------------------------
% Figure packages
\usepackage{float}
\usepackage{adjustbox}
% Text, input, formatting, and language-related packages
\usepackage[T1]{fontenc}
\usepackage{subcaption}
% \usepackage[utf8]{inputenc}
% \usepackage[nomath]{lmodern}

% Margin and formatting specifications
%\usepackage[authoryear]{natbib}
\usepackage[sort]{natbib}
\setcitestyle{square,numbers}

 %\bibliographystyle{cas-model2-names}

\usepackage{setspace}
\usepackage{subfiles} % Best loaded last in the preamble

% \usepackage[authoryear,longnamesfirst]{natbib}

% Math packages
\usepackage{amsmath, amsthm, amssymb, amsfonts, bm, nccmath, mathdots, mathtools, bigints, ulem}

\usepackage{tikz}
\usetikzlibrary{shapes.geometric,angles,quotes,calc}

% --------------------------------------------------------------------
% Packages Configurations
\usepackage{enumitem}
% --------------------------------------------------------------------
% (General) General configurations and fixes
\AtBeginDocument{\setlength{\FullWidth}{\textwidth}}	% Solves els-cas caption positioning issue
\setlength{\parindent}{20pt}
%\doublespacing
% --------------------------------------------------------------------
% Other Definitions
% --------------------------------------------------------------------
\graphicspath{{Figures/}}
% --------------------------------------------------------------------
% Environments
% --------------------------------------------------------------------
% ...

% --------------------------------------------------------------------
% Commands
% --------------------------------------------------------------------

% ==============================================================
% ========================== DOCUMENT ==========================
% ==============================================================
\begin{document} 
%  --------------------------------------------------------------------

% ===================================================
% METADATA
% ===================================================
\title[mode=title]{Parameter estimation}                      
\shorttitle{Parameter estimation}

\shortauthors{A, B, C}

\author[1]{Oliwer Sliczniuk*}[orcid=0000-0003-2593-5956]
\ead{oliwer.sliczniuk@aalto.fi}
\cormark[1]
%\credit{a}

\author[1]{Pekka Oinas}[orcid=0000-0002-0183-5558]
%\credit{b}

\author[1]{Francesco Corona}[orcid=0000-0002-3615-1359]
%\credit{c}

\address[1]{Aalto University, School of Chemical Engineering, Espoo, 02150, Finland}
%\address[2]{2}

\cortext[cor1]{Corresponding author}

% ===================================================
% ABSTRACT
% ===================================================
\begin{abstract}
Given a system of partial differential equations, $F(t,x,\dot{x},p,u)=0$, where $x$ represents state variables, $p$ are the parameters, and $u$ are control variables, the process model is simultaneously solved for both $x_i$ and a set of sensitivity functions, $dx_i/dp_j$, overall times $t$.These sensitivity functions measure the influence of the parameter change on the model's output. As an example, the supercritical extraction process is presented. The impact of mass flow rate, pressure, and inlet temperature on the model's output is discussed. The sensitivity analysis results prove that the considered variables can affect the extraction's yield and be used as control variables in optimization problems. Moreover, the local sensitivity analysis results are analyzed from a phenomenological point of view to enhance understanding of the process model.

\end{abstract}

\begin{keywords}
Supercritical extraction \sep Parameter estimation \sep Mathematical modelling
\end{keywords}

% ===================================================
% TITLE
% ===================================================
\maketitle

% ===================================================
% Section: Introduction
% ===================================================\section{Introduction}

\section{Introduction}
%\subfile{Sections/introduction}

\section{Materials and methods} \label{CH: Materials and methods}
%\subfile{Sections/Materials_and_methods}

\subsection{Extraction model} \label{CH: Extraction_model}
\subfile{Sections/Model}

%\newpage
%\section{Bayes theorem} \label{CH: Bayes}
%\subfile{Sections/Bayes_Theorem}
%\clearpage

\subsection{Parameter estimation} \label{CH: Parameter_estimation}
%\subfile{Sections/Parameter_estimation}

\section{Results} \label{CH: Results}

\section{Conclusions}

% ===================================================
% Bibliography
% ===================================================
%% Loading bibliography style file
\clearpage
%\bibliographystyle{model1-num-names}
\bibliographystyle{unsrtnat}
\bibliography{mybibfile}

\clearpage \appendix \label{appendix}
\section{Appendix} 
\subsection{Governing equations}
\subsubsection{Mass continuity}
\subfile{Sections/Gouverning_equation_derivation}

\newpage
\begin{table*}[p]
		\caption{Notation}
		\label{tab::symbols}
		\begin{tabular}{ |c|l|c| } 
			\hline
			Symbol 		& 	Description 							& Unit 						\\ \hline
			$A$			&	cross-section							& $m^2$ 					\\ \hline
			$c$			&	concentration in fluid phase			& $kg ~ m^{-3}$				\\ \hline
			$Cp$		&	specific heat of the fluid				& $J ~ mol^{-1} ~ K^{-1}$ 	\\ \hline
			$Cp_s$		&	specific heat of the solid				& $J ~ mol^{-1} ~ K^{-1}$ 	\\ \hline
			$D_e^M$		&	axial mass diffusion coefficient		& $m^2 ~ s^{-1}$			\\ \hline
			$D_e^T$		&	axial heat diffusion coefficient		& $m^2 ~ s^{-1}$			\\ \hline
			$Di$		&	internal diffusion coefficient			& $m^2 ~ s^{-1}$			\\ \hline
			$dp$		&	particle diameter						& $m$						\\ \hline
			$F(t)$		&	mass flow-rate							& $kg ~ s^{-1}$				\\ \hline
			$km$		&	partition coefficient					& $[-]$						\\ \hline
			$k^T$		&	thermal conductivity					& $W ~ m^{-1} ~ K^{-1}$		\\ \hline
			$l$			&	characteristic dimension				& $m$						\\ \hline
			$L$			&	total length of the bed					& $m$						\\ \hline
			$m$			&	mass of the oil in solid phase			& $kg$						\\ \hline
			$m_0$		&	initial mass of the oil in solid phase	& $kg$						\\ \hline
			$M_{CO_2}$	&	molecular mass of CO2					& $mol ~ kg^{-1}$			\\ \hline
			$Np$		&	number of model parameters and control variables & $[-]$			\\ \hline
			$N_{\theta}$&	number of model parameters				& $[-]$						\\ \hline
			$Nu$		&	number of control variables				& $[-]$						\\ \hline
			$Nz$		&	number of grid points in z-direction	& $[-]$						\\ \hline
			$p$			&	vector of model parameters and control variables	& $[-]$			\\ \hline
			$P(t)$		&	pressure								& $bar$						\\ \hline
			$Pe$		&	Peclet's number							& $[-]$						\\ \hline
			$q$			&	concentration in solid phase			& $kg ~ m^{-3}$				\\ \hline
			$R$			&	gas constant							& $J ~ K^{-1} ~ mol^{-1}$	\\ \hline
			$Re$		&	Reynolds number							& $[-]$						\\ \hline
			$t$			&	time									& $s$						\\ \hline
			$T$			&	temperature								& $K$						\\ \hline
			$T_0$		&	initial temperature						& $K$						\\ \hline
			$V$			&	volume of the extractor					& $m^3$						\\ \hline
			$y$			&	yield	 								& $[-]$						\\ \hline
			$z$			&	length									& $m$						\\ \hline
			$Z$			&	compressibility	factor					& $[-]$						\\ \hline
			$\epsilon$	&	void fraction							& $[-]$						\\ \hline
			$\rho$		&	density of the fluid					& $kg ~ m^{-3}$				\\ \hline
			$\rho_s$	&	solid density							& $kg ~ m^{-3}$				\\ \hline
			$\mu$		&	shape coefficient						& $[-]$						\\ \hline
			$\theta$	&	vector of model parameters				& $[-]$						\\ \hline
			$\eta$		&	viscosity								& $cP$						\\ \hline				
						& 											&							\\ \hline			
		 	Subscript	& 											&							\\ \hline
			$0$			&	initial conditions						& $[-]$						\\ \hline
			$^*$		&	equilibrium conditions					& $[-]$						\\ \hline							
		\end{tabular}
\end{table*}
\end{document}