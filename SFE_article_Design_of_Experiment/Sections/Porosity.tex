\documentclass[../Article_Design_of_Experiment.tex]{subfiles}
\graphicspath{{\subfix{../Figures/}}}
\begin{document}
	
	To calculate the porosity of the solid particles $\phi$, we can use the following formula:

    {\footnotesize
        \begin{equation}
            \phi = \dfrac{\text{Volume with pores} - \text{Volume without pores}}{\text{Volume with pores}}
        \end{equation}
    }

    We can first calculate the volume, excluding the pore space, using the true density of the material:

    {\footnotesize
        \begin{equation}
            \text{Volume excluding pores} = \dfrac{\text{mass}}{\text{true density}} = \dfrac{1\ \text{kg}}{1256.58\ \text{kg/m}^3} 0.0007958\ \text{m}^3
        \end{equation}
    }
    
    We can then use the volume of the solid particles, including the pore space and the volume excluding the pore space, to calculate the porosity:

    {\footnotesize
        \begin{equation}
            \phi = \dfrac{\text{Volume with pores} - \text{Volume without pores}}{\text{Volume with pores}}   
        \end{equation}
    }
    
    Since 1 kg of solid particles occupies 1.6 L, which is equivalent to 0.0016 m³, we have:

    {\footnotesize
        \begin{equation}
            \phi = \dfrac{0.0016\ \text{m}^3 - 0.0007958\ \text{m}^3}{0.0016\ \text{m}^3} = 0.5026
        \end{equation}
    }
    
    Therefore, the porosity of the solid particles is 0.5026 or 50.26\%. This means that half of the total volume of the solid particles is made up of pore space.
	
\end{document}