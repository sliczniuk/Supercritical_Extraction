\documentclass[../Article_Design_of_Experiment.tex]{subfiles}
\graphicspath{{\subfix{../Figures/}}}
\begin{document}
	
	The design of experiments (DOE) also known as experiment design or experimental design, is the design of any task that aims to describe and explain the variation of information under conditions that are hypothesized to reflect the variation. The term is generally associated with experiments in which the design introduces conditions that directly affect the variation, but may also refer to the design of quasi-experiments, in which natural conditions that influence the variation are selected for observation.
		
	\citet{Oinas1992} presents how the experimental design methods can be applied for analyzing mass transfer and reaction in stirred gas—liquid and gas—liquid—solid reactors. In this work a sequential approach to experimentation is suggested. First preliminary set of 8 2-level experiments ($2^8$ factorial design) were performed. The identifiability of the parameters was then considered and some reparametrisations done. The precision of the parameter values were further improved with D-optimal experiments. 
	
	\citet{Kassama2008} used the response surface methodology with the central composite rotatetable design (CCRD) model. The goal of this work was to optimize parameters for supercritical carbon dioxide extraction of lycopene ($C_{40}H_{56}$) from dried tomato skin. The CCRD consisting of three-factored factorial design with two levels was used in this study. The factors used were temperature of the extraction chamber (40 and 70 $^\circ C$), pressure of the extraction fluid (25 and 45 MPa), and modifier concentration (5 and 15\%). A second-degree polynomial equation was developed from a response surface analysis for all trans-lycopene yield and the highest yield was predicted at 62 $^\circ C$, 45 MPa (450 bar) and 14\% temperature, pressure and modifier concentration, respectively and the recovery of all trans-lycopene was 33\%.
	
	\citet{Abrahamsson2016} selected extraction conditions based on a symmetrical full factorial design of experiments with three factors and three levels. The factors were pressure ranging from 15 to 30 MPa, temperature from 313 to 353 K and compressed $CO_2$ pump volumetric flow rate from 0.5 to 1.5 mL/min, with three replicates at the center point (22.5 MPa, 333 K, 1 mL/min) thus resulting in a total of 29 extractions. By including all of the 29 experiments for the model calibration, a more robust estimates of parameters were obtained.
	
\end{document}