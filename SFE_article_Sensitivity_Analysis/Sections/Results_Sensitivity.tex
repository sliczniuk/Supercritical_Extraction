\documentclass[../Article_Model_Parameters.tex]{subfiles}
\graphicspath{{\subfix{../Figures/}}}
\begin{document}
	
	\label{CH: Results}
	
	This work investigates the influence of inlet temperature, pressure, and mass flow rate on the state space and the extraction yield. The process model and parameters have been discussed in {\color{red}article 1}. The process model was calibrated on the set of experiments obtained at different operating conditions, $40$ - $50^\circ C$ and 200 - 300 bar. The sensitivity analysis has been performed under the assumption that the system operates are $45^\circ C$, 250 bar and 0.4 l/min. 
	
	\subsection{Flow rate}
	
	The increase in the mass-flow rate affects the system simultaneously along the spatial direction by increasing the velocity but without affecting the thermodynamic state of the fluid. As a result, Figure \ref{fig:Sensitivty_F_P} shows no change in pressure during the simulation. 
    
    \begin{figure}[h!]
    	\centering
    	\includegraphics[trim = 1.5cm 1cm 0.0cm 1.0cm,clip,width=\columnwidth]{/Results_sensitivity/P_F.pdf}
    	\caption{The effect of $F$ change on $P$}
    	\label{fig:Sensitivty_F_P}
    \end{figure}
    
    It's important to note that $h$ represents enthalpy but not total enthalpy, thus excluding kinetic energy contribution. As a consequence of the modelling assumptions, changes in $h$ and $\rho_f$ occur only in response to direct change in pressure or temperature, which is demonstrated by no significant deviation of $h \times \rho_f$ in Figure \ref{fig:Sensitivty_F_H}.
    
    %When utilizing Dirichlet boundary conditions, it's crucial to be aware of potential numerical artefacts from the differing methods used to compute them. The system's enthalpy is determined through the time evolution of governing equations, while the inlet's enthalpy depends on the inlet temperature and pressure, which are the controls. A minor numerical mismatch between these values may manifest as an enthalpy difference propagating along the spatial domain. To ensure consistency between the fluid at the inlet and inside the computational domain, this analysis employs Neumann boundary conditions.
        
    \begin{figure}[h!]
    	\centering
    	\includegraphics[trim = 1.5cm 1cm 0.0cm 1.0cm,clip,width=\columnwidth]{/Results_sensitivity/H_F.pdf}
    	\caption{The effect of $F$ change on $\rho \times h$}
    	\label{fig:Sensitivty_F_H}
    \end{figure}
   
   Figure \ref{fig:Sensitivty_F_CS} shows how the flow rate affects the concentration of the solute in the solid phase. At the beginning of the extraction process, the flow rate have a low impact on the extraction process due to the dominance of the concentration gradient in the kinetic regime. The corresponding sensitivities are close to zero. As time progresses, the increase in the mass flow rate have greater influence on the extraction kinetics, causing the sensitivities to decrease towards their minimum values. Negative sensitivities indicate a faster extraction rate. Eventually, the amount of solute in the solid phase decreases, and the extraction kinetic becomes limited by the concentration gradient. This behaviour is represented by sensitivities, which increase from their minimal values and asymptotically approach to zero. The asymptotic movement of sensitivities can be explained by the fact that an increase in the flow rate does not impact the extraction process when all the solute has been removed from the solid phase.
    
    \begin{figure}[h!]
    	\centering
    	\includegraphics[trim = 1.5cm 1cm 0.0cm 1.0cm,clip,width=\columnwidth]{/Results_sensitivity/CS_F.pdf}
    	\caption{The effect of $F$ change on $C_s$}
    	\label{fig:Sensitivty_F_CS}
    \end{figure}
    
    Figure \ref{fig:Sensitivty_F_CF} illustrates how the concentration of solute in the fluid phase responds to an increase in the flow rate. Initially, sensitivities are close to zero, indicating a minimal system response. The growth in flow rate affects $C_f(z,t)$ indirectly by increasing the velocity and consequently elevating the concentration gradient. As a result, positive sensitivities emerge within the system, forming a front that progresses in the direction of flow. The positive front indicates that the larger amount of solute moves faster across the system. When the amount of the solute in the solid phase becomes a limiting factor, then the concentration gradient diminishes and slows down the extraction kinetics. The corresponding sensitivities form a front composed of negative sensitivities propagating through the extractor. The negative front indicates that the solute concentration in the fluid phase becomes lower than before the flow rate increment. Eventually, the negative sensitivities start to increase and asymptotically approach zero.
    
    \begin{figure}[h!]
    	\centering
    	\includegraphics[trim = 1.5cm 1cm 0.0cm 1.0cm,clip,width=\columnwidth]{/Results_sensitivity/CF_F.pdf}
    	\caption{The effect of $F$ change on $C_f$}
    	\label{fig:Sensitivty_F_CF}
    \end{figure}

    Figure \ref{fig:Sensitivty_F_y} illustrates how the increase in flow rate affects the extraction yield. Initially, the sensitivity curve remains flat. This occurs because the fixed bed doesn't occupy the entire volume of the extractor. Hence, some time is required for the fluid to flow through the empty portion of the extractor to reach its outlet. Only when the solute in the fluid phase reaches the extractor's outlet can the yield be measured. After the idle time, the system response can be observed as the increment of  $dy/dF$. The positive sensitivity value indicates improved process efficiency, increasing yield. As time progresses, the sensitivity reaches its maximum and diminishes due to a decreasing concentration gradient. Eventually, $dy/dF$ asymptotically approaches zero. This happens because the amount of solute in the fluid phase becomes a limiting factor, and the process is in the diffusion regime. %The simulation time was extended to demonstrate the convergence of $dy/dF$ toward zero.
    
    \begin{figure}[h!]
    	\centering
    	\includegraphics[trim = 1.5cm 1cm 0.0cm 1.0cm,clip,width=\columnwidth]{/Results_sensitivity/Y_F.pdf}
    	\caption{The effect of $F$ change on $y(t)$}
    	\label{fig:Sensitivty_F_y}
    \end{figure}
    
    \subsection{Pressure}
    
    As discussed in Chapter \ref{CH:Governing_equations_chapter}, a small pressure wave propagates at the speed of sound relative to the flow. If the flow velocity is relatively low, all pressure changes are hydrodynamic (resulting from velocity motion) rather than thermodynamic. The Low Mach-number assumption leads to instant propagation of the thermodynamic pressure throughout the system. This assumption allows considering a single pressure value for the entire system, as all changes occur simultaneously within the machine. Figure \ref{fig:Sensitivty_P_P} illustrates a step function representing the pressure change in the system.
    
    \begin{figure}[h!]
    	\centering
    	\includegraphics[trim = 1.5cm 1cm 0.0cm 1.0cm,clip,width=\columnwidth]{/Results_sensitivity/P_P.pdf}
    	\caption{The effect of $P$ change on $P$ in the system}
    	\label{fig:Sensitivty_P_P}
    \end{figure}
    
	According to Equation \ref{EQ:Enthalpy_equation}, the pressure change directly affects the quantity $h \times \rho_f$ through $\partial (P(t) A_f) / \partial t$, leading to the step change along the whole system, as presented in Figure \ref{fig:Sensitivty_P_H}. Depending on the configuration of the system, two cases are possible. If Dirichlet boundary conditions are applied, the inlet temperature is maintained at the predefined value and may differ from the temperature in the extractor. In such a case, the temperature difference will cause the heat front to propagate through the system. Alternatively, Neumann boundary conditions can be applied to ensure that the temperature inside the extractor matches that at its inlet. In this work, the second approach was chosen.
    
    \begin{figure}[h!]
    	\centering
    	\includegraphics[trim = 1.5cm 1cm 0.0cm 1.0cm,clip,width=\columnwidth]{/Results_sensitivity/H_P.pdf}
    	\caption{The effect of $P$ change on $(h \times \rho_f)$ in the system}
    	\label{fig:Sensitivty_P_H}
    \end{figure}

	The pressure change affects the mass transfer in two ways. As given in Chapter \ref{CH: Continuity}, the velocity is inversely proportional to the density; hence, the higher density of the fluid leads to a lower velocity and larger residence time. The second way is given by the correlations, which connect the pressure and the extraction kinetic parameters as presented in {\color{red}article 1}, the $D_i^R$ increases with the fluid density, which leads to a higher extraction rate. The cumulative effect of the pressure change can be observed in Figure \ref{fig:Sensitivty_P_CS}. The sensitivity plot shows a close-to-uniform decay of sensitivities along the fixed bed. The negative values of sensitivities suggest a faster extraction rate. No matter the location of the sensitivity measure in the bed, the general behaviour stays the same. Every sensitivity starts with zero and decreases to a minimum value. After the extremum, the sensitivities asymptotically increase to zero.

	\begin{figure}[h!]
		\centering
		\includegraphics[trim = 1.5cm 1cm 0.0cm 1.0cm,clip,width=\columnwidth]{/Results_sensitivity/CS_P.pdf}
		\caption{The effect of $P$ change on $C_s$}
		\label{fig:Sensitivty_P_CS}
	\end{figure}

	The deviation in the solute concentration in the fluid phase caused by the pressure change is presented in Figure \ref{fig:Sensitivty_P_CF}. As discussed earlier, the pressure change directly influences the extraction kinetics parameters. Figure \ref{fig:Sensitivty_P_CS} is characterized by negative sensitivities, suggesting a higher extraction rate due to the pressure change. Consequently, an increase in solute concentration in the fluid phase is expected. This increase in solute concentration in the fluid phase is visible in Figure \ref{fig:Sensitivty_P_CF} as positive sensitivities, forming a front that moves along the extractor. As the extraction process accelerates, more solute enters the fluid phase, explaining the "hot spot" in the Figure. Subsequently, the solute flows through the system without solid particles, and the diffusion effect becomes noticeable. After the front of positive sensitivities, a small front of negative sensitivities can be observed. Due to the higher extraction rate, more solute is extracted at the beginning of the extraction process, resulting in less solute being available for extraction later. This effect is reflected in Figure \ref{fig:Sensitivty_P_CF} as negative sensitivities. Eventually, the negative sensitivities approach zero.

	\begin{figure}[h!]
		\centering
		\includegraphics[trim = 1.5cm 1cm 0.0cm 1.0cm,clip,width=\columnwidth]{/Results_sensitivity/CF_P.pdf}
		\caption{The effect of $P$ change on $C_f$}
		\label{fig:Sensitivty_P_CF}
	\end{figure}

	The impact of pressure increase on extraction yield is depicted in Figure \ref{fig:Sensitivty_P_y}. The initial flat curve reflects a system delay caused by the empty space within the extractor that the fluid phase needs to flow through. The first observed deviation is a small negative sensitivity, which can be related to a lower fluid phase velocity. Keeping the same mass flow rate and increasing the density (by increasing the pressure) decreases the velocity. Next, the sensitivity rises, which indicates an increase in yield. $dy / dP$ reaches its positive maximum and then declines due to a limited amount of solute remaining in the solid phase. The sensitivity curve draws into negative values, reach a minimum point, and converges to zero. The simulation duration was extended to demonstrate the convergence of $dy / dP$ towards zero.

	\begin{figure}[h!]
		\centering
		\includegraphics[trim = 1.5cm 1cm 0.0cm 1.0cm,clip,width=\columnwidth]{/Results_sensitivity/Y_P.pdf}
		\caption{The effect of $P$ change on $y(t)$}
		\label{fig:Sensitivty_P_y}
	\end{figure}

	\subsection{Inlet temperature}
	
	The sensitivity analysis of the inlet temperature differs from the two cases presented earlier because the perturbation does not affect the entire system instantaneously; instead, it propagates through the system. As the fluid with the modified temperature flows along the system, it gradually modifies the mass transfer parameters. One important assumption is that the inlet temperature does not affect the pressure, and as a result, a horizontal line is present in Figure \ref{fig:Sensitivty_P_T}.
	
	\begin{figure}[h!]
		\centering
		\includegraphics[trim = 1.5cm 1cm 0.0cm 1.0cm,clip,width=\columnwidth]{/Results_sensitivity/P_T_{in}.pdf}
		\caption{The effect of $T_{in}$ change on $P$ in the system}
		\label{fig:Sensitivty_P_T}
	\end{figure}

	The heat front propagation is presented in Figure \ref{fig:Sensitivty_T_H}. The initial system had constant temperature along the whole spatial domain. The inlet temperature is a control represented by the Dirichlet boundary condition. The inlet $(h\times \rho_f)$ inside the spatial domain is calculated based on the initial inlet temperature and given pressure. Any deviation in $T_{in}$ affects $(h\times \rho_f)$ at the inlet, propagating according to the governing equations.
	
	\begin{figure}[h!]
		\centering
		\includegraphics[trim = 1.5cm 1cm 0.0cm 1.0cm,clip,width=\columnwidth]{/Results_sensitivity/H_T_{in}.pdf}
		\caption{The effect of $T_{in}$ change on $(h \times \rho_f)$ in the system}
		\label{fig:Sensitivty_T_H}
	\end{figure}

	Figure \ref{fig:Sensitivty_T_CS} illustrates how the change in inlet temperature affects the concentration of solute in the solid phase. As presented in {\color{red}article 1}, the value of $D_i^R$ decreases as density decreases. Therefore, it is expected to observe positive sensitivities in Figure \ref{fig:Sensitivty_T_CS}, indicating a slower extraction rate. Initially, the sensitivities are zero along the fixed bed because the heat front requires time to propagate to the fixed bed. Since this propagation is not instantaneous, a non-uniform distribution of sensitivities along the fixed bed becomes evident. All the sensitivities gradually increase until they reach their maxima. When the concentration gradient becomes the limiting factor, the sensitivities start to decrease.

	\begin{figure}[h!]
		\centering
		\includegraphics[trim = 1.5cm 1cm 0.0cm 1.0cm,clip,width=\columnwidth]{/Results_sensitivity/CS_T_{in}.pdf}
		\caption{The effect of $T_{in}$ change on $C_s$ in the system}
		\label{fig:Sensitivty_T_CS}
	\end{figure}

	The influence of inlet temperature on solute concentration in the fluid phase is depicted in Figure \ref{fig:Sensitivty_T_CF}. Initially, all the sensitivities remain at zero due to the idle period. As the fluid with elevated temperature flows through the fixed bed, the internal mass transfer slows down, resulting in negative sensitivities. The diffusion effect is observable in the empty section, where the front composed of sensitivities becomes blurry. After reaching their minima, the sensitivities increase and reach slightly positive values. This behaviour can be explained by considering that the heat front slowed mass transfer, causing more solute to remain in the solid phase.
	
	\begin{figure}[h!]
		\centering
		\includegraphics[trim = 1.5cm 1cm 0.0cm 1.0cm,clip,width=\columnwidth]{/Results_sensitivity/CF_T_{in}.pdf}
		\caption{The effect of $T_{in}$ change on $C_f$ in the system}
		\label{fig:Sensitivty_T_CF}
	\end{figure}

	Figure \ref{fig:Sensitivty_T_y} depicts how an increase in inlet temperature alters the extraction yield. Initially, the sensitivity curve remains flat. This is because the fixed bed doesn't occupy the entire volume of the extractor, and the fluid requires some time to flow through the empty portion of the extractor. The first observed response of the system is a small increment of the $dy/dT_{in}$ caused by an increment of the velocity( which is inversely proportional to the fluid density). After the small positive peak, the sensitivity curve begins to decrease. The negative value of the sensitivity indicates a lower process efficiency. Over time, the sensitivity reaches its minimum and then increases due to a higher concentration gradient than in the case without the disturbance. Eventually, the $dy/dT_{in}$ curve flattens around a negative value. The flattening of the yield curve suggests that the mass transfer parameters limit the extraction rate, and the residual solute in the solid phase becomes difficult to obtain. The simulation time was extend to show how the sensitivity plot flatten.

	\begin{figure}[h!]
		\centering
		\includegraphics[trim = 1.5cm 1cm 0.0cm 1.0cm,clip,width=\columnwidth]{/Results_sensitivity/Y_T_{in}.pdf}
		\caption{The effect of $T_{in}$ change on $y(t)$ in the system}
		\label{fig:Sensitivty_T_y}
	\end{figure}
	
\end{document}


































