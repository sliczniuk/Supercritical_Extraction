\documentclass[../Article_Model_Parameters.tex]{subfiles}
\graphicspath{{\subfix{../Figures/}}}
\begin{document}
	
	\label{CH: Results}
	
	The sensitivity equations were solved simultaneously with the original process model. The focus of this work is to investigate the influence of inlet temperature, pressure and mass flow-rate on the state-space as wall as on the extraction yield. The process model and parameters have been discussed in {\color{red}article 1}. The process model was calibrated on set of experiments obtained at different operating conditions, $40^\circ C$ - $50^\circ C$ and 200 bar - 300 bar. The sensitivity analysis has been performed under assumption that the system operates are $45^\circ C$ and 250 bar.
	
	\subsection{Flow-rate}
	
	The increase of the mass-flow rate affects the whole system simultaneously in the spatial direction. The change in mass flow-rate makes the fluid moves faster, but its thermodynamic state is not affected. As the result, Figure \ref{fig:Sensitivty_F_P} show no change in pressure during the simulation. 
    
    \begin{figure}[h!]
    	\centering
    	\includegraphics[trim = 2cm 0cm 0cm 0cm,clip,width=\columnwidth]{/Results_sensitivity/P_F.pdf}
    	\caption{The effect of $F$ change on $P$}
    	\label{fig:Sensitivty_F_P}
    \end{figure}
    
    Similarly, the energy in the system (defined as $\rho \times h$) is not affected as presented on Figure \ref{fig:Sensitivty_F_H}. As pressure and temperature are not affected by the increase in the mass flow-rate, the fluid density $\rho$ and enthalpy $h$ are insensitive to flow-rate change.
    
    \begin{figure}[h!]
    	\centering
    	\includegraphics[trim = 2cm 0cm 0cm 0cm,clip,width=\columnwidth]{/Results_sensitivity/H_F.pdf}
    	\caption{The effect of $F$ change on $\rho \times h$}
    	\label{fig:Sensitivty_F_H}
    \end{figure}
    
    The increase in the mass flow rate affects the concentration gradient and accelerates the extraction kinetics. Negative sensitivities in Figure \ref{fig:Sensitivty_F_CS} present the faster extraction rate. Negative values of $dc_s/dF$ can be interpreted as a 'faster' mass loss from the solid phase, which means the extraction efficiency improves. At the beginning of the extraction process, the change in the flow rate has a minimal effect on the extraction process (represented by close to zero sensitivities), due to the high concentration gradient, which leads the extraction kinetic. Later, the increment of the mass flow rate accelerates the extraction kinetic and sensitivities along the decrease to minimum values. Over time, the amount of solute in the solid phase decreases, and eventually, the extraction kinetic becomes limited by the concentration gradient. This behaviour is represented by sensitivities, which asymptotically go to zero. The asymptotic movement of sensitivities can be explained by the fact that the increase in the flow rate does not affect the extraction if all the solute has been removed from the solid phase.
    
    \begin{figure}[h!]
    	\centering
    	\includegraphics[trim = 2cm 0cm 0cm 0cm,clip,width=\columnwidth]{/Results_sensitivity/CS_F.pdf}
    	\caption{The effect of $F$ change on $C_s$}
    	\label{fig:Sensitivty_F_CS}
    \end{figure}
    
    Figure \ref{fig:Sensitivty_F_CF} shows how the concentration of solute in the fluid phase is affected by the flow-rate increment. Initially, sensitivities are close to zero, which indicates very little system response. The flow-rate growth affects the ${\color{blue}C_f}(z,t)$ by increasing the velocity of the solute across the system, which also increases the concentration gradient. As a result, the positive sensitivities appear in the system and form a front, which moves from the inlet to the outlet of the extractor. Because the larger amount of solute moves faster across the system, the total amount of solute in both phases decreases faster, which eventually causes negative sensitivities to appear in the system. The negative sensitivities form the second front, which moves across the extractor. Eventually, the negative sensitivities asymptotically go to zero.
    
    \begin{figure}[h!]
    	\centering
    	\includegraphics[trim = 2cm 0cm 0cm 0cm,clip,width=\columnwidth]{/Results_sensitivity/CF_F.pdf}
    	\caption{The effect of $F$ change on $C_f$}
    	\label{fig:Sensitivty_F_CF}
    \end{figure}
    
    Figure \ref{fig:Sensitivty_F_y} shows how the flow-rate increment changes the extraction yield. Initially, the sensitivity curve stays flat, which is caused by the fact that the fixed bed does not occupy the whole volume of the extractor, and the fluid needs some time to flow through the empty part of the extractor to reach its outlet. When the solute in the fluid phase reaches the outlet of the extractor, then system response can be observed. The $dy/dF$ starts to increase, and positive value of sensitivity indicates an improvement in the process efficiency, hence the yield increment. Over time, the sensitivity reaches its maximum and diminishes due to decreasing concentration gradient. The $dy/dF$ asymptotically goes to zero because the amount of solute in the fluid phase becomes a limiting factor of the extraction process, and the flow-rate increment has less effect on the extraction yield.
    
    \begin{figure}[h!]
    	\centering
    	\includegraphics[trim = 2cm 0cm 0cm 0cm,clip,width=\columnwidth]{/Results_sensitivity/Y_F.pdf}
    	\caption{The effect of $F$ change on $y(t)$}
    	\label{fig:Sensitivty_F_y}
    \end{figure}
    
    \subsection{Pressure}
    
    As discussed in Chapter \ref{CH:Governing_equations_chapter}, a small pressure wave propagates with the speed of sound relative to the flow. If the flow velocity is relatively low, all pressure changes are hydrodynamic (due to velocity motion) rather than thermodynamic which leads to $\partial {\color{blue}P}/\partial {\color{orange}\rho_f} = \infty$. The Low Mach-number assumption leads to instant pressure propagation along the system. It allows to consider one value of pressure for the whole system as all the changes occur in the machine simultaneously. Figure \ref{fig:Sensitivty_P_P} represents a step function, which indicate the pressure change in the system. 
    
    \begin{figure}[h!]
    	\centering
    	\includegraphics[trim = 2cm 0cm 0cm 0cm,clip,width=\columnwidth]{/Results_sensitivity/P_P.pdf}
    	\caption{The effect of $P$ change on $P$ in the system}
    	\label{fig:Sensitivty_P_P}
    \end{figure}
    
	As a result of the Low Mach-number assumption and the pressure change, the temperature and density changes happen along the system simultaneously. As presented by Equation \ref{EQ:Enthalpy_equation}, the pressure change affect the quantity $h \times \rho_f$ directly through $\frac{\partial ({\color{blue}P}(t) A_f)}{\partial t}$, which leads to a step-change presented in Figure \ref{fig:Sensitivty_P_H}. Depending on the configuration of the system, two cases are possible. As the total energy in the extractor has changed, there might be a difference between the fluid inside the equipment and the inlet temperature(defined as a boundary condition). If the Dirichlet boundary condition is applied, the inlet temperature is kept at the pre-defined value and might differ from the temperature in the extractor. In such a case, the temperature difference will cause the heat front to propagate along the system. Alternatively, the Neumann boundary condition can be applied to ensure that the temperature inside the extractor and at its inlet is the same. In the case of this work, the second approach was chosen to simplify the discussion.
    
    \begin{figure}[h!]
    	\centering
    	\includegraphics[trim = 2cm 0cm 0cm 0cm,clip,width=\columnwidth]{/Results_sensitivity/H_P.pdf}
    	\caption{The effect of $P$ change on $(h \times \rho_f)$ in the system}
    	\label{fig:Sensitivty_P_H}
    \end{figure}

	The pressure change affects the mass transfer in two ways. As given in Chapter \ref{CH: Continuity}, the velocity is inversely proportional to the density, hence the higher density of the fluid leads to the lower velocity and the residence time. The second way is given by the relationships, which connect the pressure and the extraction kinetic term. As presented in {\color{red}article 1}, the $D_i^R$ increase with the fluid density, which leads to higher extraction rate. The cumulative effect of the pressure change can be observed in Figure \ref{fig:Sensitivty_P_CS}. The 

	\begin{figure}[h!]
		\centering
		\includegraphics[trim = 2cm 0cm 0cm 0cm,clip,width=\columnwidth]{/Results_sensitivity/CS_P.pdf}
		\caption{The effect of $P$ change on $C_s$}
		\label{fig:Sensitivty_P_CS}
	\end{figure}
	
\end{document}


































