\documentclass[../Article_Model_Parameters.tex]{subfiles}
\graphicspath{{\subfix{../Figures/}}}
\begin{document}
	
	Sensitivity analysis is the study of how the uncertainty in the output of a mathematical model or system (numerical or otherwise) can be divided and allocated to different sources of uncertainty in its inputs or model parameters. As a mathematical model can be complex and consists of multiple parameters, a method to evaluate the relation between inputs and outputs, such a technique is called sensitivity analysis. There are a large number of approaches to performing a sensitivity analysis, which include but it is not limited to the following methods:
	
	\begin{itemize}
		\item One-at-a-time method
		\item Derivative-based local methods
		\item Variance-based methods
	\end{itemize}
	
	\citet{Fiori_2007} applied the sensitivity analysis to supercritical extraction process. The sensitivity calculations were performed by varying the parameters within their confidence interval and observing how the model results change, which allows to establish the effect of the uncertainties on model predictions. The sensitivity analysis revealed that two parameters are particularly important for the extraction process: the particle diameter and the internal mass transfer coefficient. The effect of changing some operative conditions is also investigated, underlining how the solvent flow rate and the seed milling affect the extraction process.
	
	\citet{Santos2000}, in their work considered a diffusive model for semi-continuous isothermal and isobaric extraction process using supercritical carbon dioxide. The parametric sensitivity analysis by means of a factorial design in two levels was considered. The model parameters were disturbed and their main effects analysed, so that it is possible to propose strategies for high performance operation. The parametric sensitivity analysis was carried out by applying disturbances of 10\% in the values of the normal operation conditions.
	
	\citet{Hatami2024} deploed two distinct strategies for sensitivity analysis. The first strategy applies uniform fluctuations to technical and economic parameters, while the second employs actual parameter ranges for fluctuation. The combination of these two strategies provide insights into both the sensitivity of net present value (NPV) to the unit relative change in each factor and the maximum potential of each factor to influence NPV.
	
	\citet{Poletto1996} provided a general dimensionless model was developed for a sensitivity analysis of the supercritical extraction process of vegetable oils and essential oils. Two dimensionless parameters, a dimensionless partition coefficient, and a dimensionless characteristic time, appeared as the most important parameters to monitor the evolution of the extraction process. The sensitivity calculations were performed by varying the parameters and analysis the model response.
	
\end{document}