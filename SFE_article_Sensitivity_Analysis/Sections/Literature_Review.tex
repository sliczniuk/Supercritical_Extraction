\documentclass[../Article_Model_Parameters.tex]{subfiles}
\graphicspath{{\subfix{../Figures/}}}
\begin{document}
	
	This study aims to analyse the influence of changes in operating conditions on the supercritical extraction process described in {\color{red}article 1}. The emphasis is put on the effect of to mass-flow rate, pressure and the inlet temperature. The relation between input and output are obtained by applying a sensitivity analysis on the process model. Sensitivity analysis is the study of how the uncertainty in the output of a mathematical model or system (numerical or otherwise) can be divided and allocated to different sources of uncertainty in its inputs or model parameters. There are many sensitivity analysis methods which include but are not limited to those listed below:
	
	\begin{itemize}
		\item One-at-a-time method
		\item Derivative-based local methods
		\item Variance-based methods
	\end{itemize}
	
	Different supercritical extraction models were analyzed using sensitivity analysis. One example of such work is \citet{Fiori_2007}. The sensitivity calculations were performed by varying the parameters within their confidence interval and observing how the model results change, which allows to establish the effect of the uncertainties on model predictions. The sensitivity analysis revealed that two parameters are particularly important for the extraction process: the particle diameter and the internal mass transfer coefficient. The effect of changing some operative conditions was also investigated, underlining how the solvent flow rate and the seed milling affect the extraction process.
	
	\citet{Santos2000}, in their work considered a model of supercritical extraction process for semi-continuous isothermal and isobaric extraction process using carbon dioxide as a solvent. The parametric sensitivity analysis was carried out by applying disturbances of 10\% in the values of the normal operation conditions.
	
	\citet{Hatami2024} used a one-factor-at-a-time sensitivity analysis to assess the response of NPV concerning variations in both technical and economic variables. Their findings shows that the most influential factors on NPV include the price of the extract, the interest rate, the dynamic time of SFE, and the project lifetime.
	
	\citet{Poletto1996} provided a general dimensionless model was developed for a sensitivity analysis of the supercritical extraction process of vegetable oils and essential oils. A dimensionless partition coefficient, and a dimensionless characteristic time, appeared as the most important parameters to monitor the evolution of the extraction process. The sensitivity calculations were performed by varying the parameters and analysis the model response.
	
\end{document}