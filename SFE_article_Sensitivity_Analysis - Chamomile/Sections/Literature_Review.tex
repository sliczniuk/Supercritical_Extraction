\documentclass[../Article_Sensitivity_Analsysis.tex]{subfiles}
\graphicspath{{\subfix{../Figures/}}}
\begin{document}
	
	This study aims to analyze the influence of changes in operating conditions on the supercritical extraction process described in {\color{red}article 1}. The emphasis is put on the effect of the mass flow rate, pressure, and the inlet temperature. The relation between input and output is obtained by applying a sensitivity analysis. Sensitivity analysis examines the impact of varying inputs or model parameters on the system's output. The aim is to understand and to allocate the source of uncertainty in the output to the corresponding inputs or parameters. There are many sensitivity analysis methods, which include but are not limited to those listed below:
	
	\begin{itemize}
		\item One-at-a-time method
		\item Derivative-based local methods
		\item Variance-based methods
	\end{itemize}
	
	Different supercritical extraction models were analyzed using sensitivity analysis. \citet{Fiori_2007}, performed the sensitivity calculations by varying the parameters within their confidence interval and observing how the model results changed. This allows to evaluate the effect of the uncertainties on model predictions. The sensitivity analysis revealed that the particle diameter and the internal mass transfer coefficient are significant for the extraction process. The effect of changing some operative conditions was also investigated, underlining how the solvent flow rate and the seed milling affect the extraction process.
	
	\citet{Santos2000}, in their work, considered a model of supercritical extraction process for semi-continuous isothermal and isobaric extraction process using carbon dioxide as a solvent. The parametric sensitivity analysis was carried out by applying disturbances of 10\% in the values of the normal operation conditions.
	
	\citet{Hatami2024} used a one-factor-at-a-time sensitivity analysis to assess the response of NPV concerning variations in both technical and economic variables. Their findings show that the most influential factors on NPV include the price of the extract, the interest rate, the dynamic time of SFE, and the project lifetime.
	
	\citet{Poletto1996} provided a general dimensionless model for the supercritical extraction process of vegetable and essential oils and applied a sensitivity analysis. They found that a dimensionless partition coefficient and a dimensionless characteristic time appeared as the most important parameters of the extraction process. The sensitivity calculations were performed by varying the parameters and analyzing the model response.
	
\end{document}